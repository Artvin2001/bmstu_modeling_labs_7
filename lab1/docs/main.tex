\documentclass[12pt]{report}
\usepackage[utf8]{inputenc}
\usepackage[russian]{babel}

\usepackage{indentfirst}
\usepackage{amsmath}
\usepackage{graphicx}
\usepackage{listings}
\usepackage{biblatex}
\usepackage{float}
\usepackage{floatrow}
\usepackage[13pt]{extsizes}

\addbibresource{report.bib}

\lstset{extendedchars=\true, frame=single, numbers=left}

\usepackage[left=3cm,right=1.5cm, top=2cm,bottom=2cm,bindingoffset=0cm]{geometry}
\usepackage{titlesec, blindtext, color}
\newcommand{\hsp}{\hspace{20pt}}
\titleformat{\chapter}[hang]{\Huge\bfseries}{\thechapter\hsp{|}\hsp}{0pt}{\Huge\bfseries}

\begin{document}
\def\chaptername{}
\thispagestyle{empty}
\begin{titlepage}
	\noindent \begin{minipage}{0.15\textwidth}
	\includegraphics[width=\linewidth]{bmstu.jpg}
	\end{minipage}
	\noindent\begin{minipage}{0.9\textwidth}\centering
		\textbf{Министерство науки и высшего образования Российской Федерации}\\
		\textbf{Федеральное государственное бюджетное образовательное учреждение высшего образования}\\
		\textbf{~~~«Московский государственный технический университет имени Н.Э.~Баумана}\\
		\textbf{(национальный исследовательский университет)»}\\
		\textbf{(МГТУ им. Н.Э.~Баумана)}
	\end{minipage}
	
	\noindent\rule{18cm}{3pt}
	\newline\newline
	\noindent ФАКУЛЬТЕТ $\underline{\text{«Информатика и системы управления»}}$ \newline\newline
	\noindent КАФЕДРА $\underline{\text{«Программное обеспечение ЭВМ и информационные технологии»}}$\newline\newline\newline\newline\newline
	
	
	\begin{center}
        \Large\textbf{Отчет по лабораторной работе №1 по курсу "Моделирование"}
    \end{center}

    \\
    %\noindent\textbf{Тема} $\underline{\text{~~~~~~~~~~~~~~~~~~~~~~~~~~~~~}}$\newline\newline\newline
    \noindent\textbf{Студент} $\underline{\text{Заборовская А. Д.~~~~~~~~~~~~~~~~~~~~~~~~~~~~~~~~~~~~}}$\newline\newline
    \noindent\textbf{Группа} $\underline{\text{ИУ7-73Б~~~~~~~~~~~~~~~~~~~~~~~~~~~~~~~~~~~~~~~~~~~~~~~~~~~}}$\newline\newline
    \noindent\textbf{Оценка} $\underline{\text{~~~~~~~~~~~~~~~~~~~~~~~~~~~~~~~~~~~~~~~~~~~~~~~~~~~~~~~~~~~~~~~}}$\newline\newline
    \noindent\textbf{Преподаватель} $\underline{\text{Рудаков И. B.~~~~~~~~~~~~~~~~~~~~~~~~~~}}$\newline

	
	\begin{center}
		\vfill
		Москва~---~\the\year
		~г.
	\end{center}
\end{titlepage}

\chapter{Задание}
Разработать программу для построения графиков функции распределения и функции плотности распределения для следующих видов распределений:
\begin{itemize}
    \item равномерного распределения;
    \item экспоненциального распределения.
\end{itemize}
Реализовать графический интерфейс, который будет давать пользователю возможность выбора вида распределения и настройки его параметров.

\chapter{Реализация}
\section{Равномерное распределение}
Равномерное распределение -- это распределение случайной вещественной величины, принимающей значения, принадлежащие некоторому промежутку конечной длины, характеризующееся тем, что плотность вероятности на этом промежутке почти всюду постоянна.\\

Обозначение равномерного распределения: X \sim R(a, b), где a, b \in R.\\

Функция распределения:
\[ F(x) =
  \begin{cases}
    0       & \quad \text{при } x < a\\
    \frac{x - a}{b - a}  & \quad \text{при } a <= x <= b\\
    1       & \quad \text{при} x > b
  \end{cases}
\]

Функция плотности распределения:
\[ f(x) =
  \begin{cases}
    0       & \quad \text{при } x <= a\\
    \frac{1}{b - a}  & \quad \text{при } a < x <= b\\
    0       & \quad \text{при} x > b
  \end{cases}
\]

На листинге 2.1 приведены функции для вычисления значений и построения функций для равномерного распределения.

\begin{lstlisting}[label=some-code,caption=Функии для равномерного распределения, language=Python]
def uni_func_pl(x, a, b):
    if x < a or x > b:
        return 0
    else:
        return 1 / (b - a)
    
def uni_func(x, a, b):
    if x <= a:
        return 0
    elif x > b:
        return 1
    else:
        return (x - a) / (b - a)

def begin_uni(entry_a, entry_b):
    try:
       a = float(entry_a.get())
       b = float(entry_b.get())
    except:
        tk.messagebox.showerror(title='Ошибка ввода', 
            message='Введите корректные данные')
        return
    
    if (a > b):
        tk.messagebox.showerror(title='Ошибка ввода', 
            message='Левая граница должна быть меньше правой')
        return
    xu1, xu2, xu3 = [], [], []
    xu = []
    yu = []
    yup1, yup2, yup3 = [], [], []
    gr1 = a - (b - a) / 2
    gr2 = b + (b - a) / 2
    i = gr1
    
    while i < gr2:
        if i < a:
            xu1.append(i)
        elif i > b:
            xu3.append(i)
        else:
            xu2.append(i)
        xu.append(i)
        i += (gr2 - gr1) / 100
    for i in range(len(xu)):
        yu.append(uni_func(xu[i], a, b))

    for i in range(len(xu1)):
        yup1.append(uni_func_pl(xu1[i], a, b))

    for i in range(len(xu2)):
        yup2.append(uni_func_pl(xu2[i], a, b))

    for i in range(len(xu3)):
        yup3.append(uni_func_pl(xu3[i], a, b))
        
    fig, ax = plt.subplots(figsize=(5, 3))
    ax.grid()
    plt.plot(xu1, yup1, color='m')
    plt.plot(xu2, yup2, color='m')
    plt.plot(xu3, yup3, color='m')
    plt.title('Функция плотности распределения')

    fig2, ax2 = plt.subplots(figsize=(5, 3))
    ax2.grid()
    plt.plot(xu, yu)
    plt.title('Функция распределения')
    
    plt.show()
\end{lstlisting}

\section{Экспоненциальное распределение}
Экспоненциальное распределение -- это абсолютно непрерывное распределение, моделирующее время между двумя последовательными свершениями одного и того же события.\\

Обозначение экспоненциального распределения: X \sim Exp(\gamma), где \gamma > 0

Функция распределения:
\[ F(x) =
  \begin{cases}
    0       & \quad \text{при } x <= 0\\
    1 - \exp^{-\gamma x}  & \quad \text{при } x > 0
  \end{cases}
\]

Функция плотности распределения:
\[ f(x) =
  \begin{cases}
    0       & \quad \text{при } x < 0\\
    \gamma \exp^{-\gamma x}  & \quad \text{при } x >= 0
  \end{cases}
\]

На листинге 2.2 приведены функции для вычисления значений и построения функций для экспоненциального распределения.

\begin{lstlisting}[label=some-code,caption=Функии для экспоненциального распределения, language=Python]
def exp_func(x, gamma):
    gamma = -gamma
    if x <= 0:
        return 0
    else:
        return 1 - exp(x * gamma)

def exp_func_pl(x, gamma):
    if x < 0:
        return 0
    else:
        return gamma * exp(x * (-gamma))

def begin_exp(entry_al):
    try:
        gamma = float(entry_al.get())
    except:
        tk.messagebox.showerror(title='Ошибка ввода', 
            message='Введите корректные данные')
        return

    if gamma <= 0:
        tk.messagebox.showerror(title='Ошибка ввода', 
            message='Параметр должен быть больше 0')
        return
    xe = []
    ye = []
    yep = []
    i = 0
    gr2 = 10 / gamma
    
    while i < gr2:
        xe.append(i)
        i += (gr2 + 1) / 100
    for i in range(len(xe)):
        ye.append(exp_func(xe[i], gamma))
        yep.append(exp_func_pl(xe[i], gamma))

    fig, ax = plt.subplots(figsize=(5, 3))
    ax.grid()
    plt.plot(xe, yep, color='m')
    plt.title('Функция плотности распределения')

    fig2, ax2 = plt.subplots(figsize=(5, 3))
    ax2.grid()
    plt.plot(xe, ye)
    plt.title('Функция распределения')
    
    plt.show()
\end{lstlisting}
\chapter{Примеры работы}

На рисунках 3.1 и 3.2 приведены примеры работы программы.

\begin{figure}[ht!]	
	\centering{
		\includegraphics[width=1\textwidth]{p2.png}}
		\caption{Пример работы для равномерного распределенния}
\end{figure}

\begin{figure}[ht!]	
	\centering{
		\includegraphics[width=1\textwidth]{p1.png}}
		\caption{Пример работы для экспоненциального распределения}
\end{figure}

\end{document}
